\documentclass[11pt, a4paper]{article}
\usepackage[utf8]{inputenc}
\usepackage[T1]{fontenc}
\usepackage{amsmath, amssymb, amsthm}
\usepackage{geometry}
\usepackage{cite}
\usepackage{booktabs}
\usepackage{url}
\usepackage[hidelinks]{hyperref}
% Page layout
\geometry{left=3cm, right=3cm, top=3cm, bottom=3cm}
% Theorem environments
\theoremstyle{plain}
\newtheorem{theorem}{Theorem}[section]
\newtheorem{proposition}[theorem]{Proposition}
\newtheorem{lemma}[theorem]{Lemma}
\newtheorem{corollary}[theorem]{Corollary}
\theoremstyle{definition}
\newtheorem{definition}[theorem]{Definition}
\newtheorem{remark}[theorem]{Remark}
\newtheorem{conjecture}[theorem]{Conjecture}
% Title
\title{\textbf{Quantitative Predictions for Prime Values of the Titan Polynomial:\\[4pt]
The Bateman--Horn Constant and Asymptotic Density}}
\author{\textbf{Ruqing Chen}\\
GUT Geoservice Inc., Montreal, Quebec\\
\texttt{ruqing@hotmail.com}}
\date{February 2026}
\begin{document}
\maketitle
\begin{abstract}
The polynomial $Q(n)=n^{47}-(n-1)^{47}$ is a degree-46 cyclotomic
norm form whose values have no prime factor less than~$283$.
While this algebraic rigidity does not alter the fundamental sieve
dimension ($\kappa=1$), it significantly impacts the local density
of prime values.
In this paper, we rigorously compute the Bateman--Horn constant
$C_Q$ associated with this sequence.
We show that the exclusion of small prime factors leads to an
Euler product enhancement of $C_Q \approx 8.68$ relative to the
logarithmic integral for degree-46 polynomials.
This large constant implies that, under the Bateman--Horn heuristic,
prime values of $Q(n)$ occur with a frequency approximately $8.7$
times that predicted for a generic polynomial of the same degree.
We provide numerical evidence supporting the asymptotic prediction
$\pi_Q(x) \sim \frac{C_Q}{46}\operatorname{Li}(x)$.
\end{abstract}

\medskip
\noindent\textbf{MSC 2020:} 11N32, 11R18, 11Y35

\noindent\textbf{Keywords:} Bateman--Horn conjecture,
cyclotomic norm form, Euler product, polynomial primes,
shielding property, prime density
\section{Introduction}
The Bateman--Horn conjecture~\cite{BatemanHorn} provides a heuristic
asymptotic formula for the distribution of prime values of
polynomials.
For a single irreducible polynomial $f(n)$ of degree~$d$, the number
of integers $n \le x$ such that $f(n)$ is prime is denoted by
$\pi_f(x)$ and is conjectured to satisfy
\[
\pi_f(x) \;\sim\; C_f \int_2^x \frac{dt}{\log f(t)}
\;\sim\; \frac{C_f}{d} \int_2^x \frac{dt}{\log t}
\;=\; \frac{C_f}{d}\,\operatorname{Li}(x),
\]
where
\[
C_f = \prod_p \frac{1-\omega_f(p)/p}{1-1/p}
\]
and $\omega_f(p)$ is the number of solutions to
$f(n) \equiv 0 \pmod{p}$.
For the Titan polynomial $Q(n) = n^{47} - (n-1)^{47}$---we adopt
this name as a convenient label, without implying any established
nomenclature---the degree is $d=46$.
The local root structure established in~\cite{ChenTitan} shows that
$\omega_Q(p) = 0$ for all primes $p < 283$.
In this paper, we compute the specific enhancement constant $C_Q$
resulting from this shielding property and compare the predictions
with numerical data in verifiable ranges.
\section{The Shielding Property and $\omega_Q(p)$}
The local root structure of $Q(n)$, established
in~\cite{ChenTitan}, is:
\begin{enumerate}
    \item If $p < 283$, then $\omega_Q(p) = 0$.
    \item If $p \ge 283$ and $p \equiv 1 \pmod{47}$, then $\omega_Q(p) = 46$.
    \item If $p \ge 283$ and $p \not\equiv 1 \pmod{47}$, then $\omega_Q(p) = 0$.
\end{enumerate}

\begin{remark}
The vanishing $\omega_Q(p)=0$ for $p<283$ arises from three
distinct mechanisms:
(i)~for $p\in\{2,3,47\}$, the condition $(p-1)\mid 46$ implies
$Q(n)\equiv 1\pmod{p}$ for all~$n$ (congruence rigidity);
(ii)~for $p=47$ specifically, this is also the ramified prime
of $\mathbb{Q}(\zeta_{47})/\mathbb{Q}$;
(iii)~for all other primes $p<283$ with $p\not\equiv 1\pmod{47}$,
the map $x\mapsto x^{47}$ is a bijection on
$(\mathbb{Z}/p\mathbb{Z})^\times$, so no nontrivial $47$-th
root of unity exists.
The smallest prime $p\equiv 1\pmod{47}$ is $283$, which is why the
shielding extends up to that threshold.
\end{remark}

This local root structure implies that for small primes, the local
density factor $\frac{1-\omega_Q(p)/p}{1-1/p}$ simplifies to
$\frac{p}{p-1} > 1$, thereby contributing a significant boost
to~$C_Q$.
\section{Computation of the Bateman--Horn Constant}
The constant $C_Q$ is given by the infinite product:
\[
C_Q = \prod_p \left( \frac{1-\omega_Q(p)/p}{1-1/p} \right).
\]
We split the product into two parts: small primes ($p < 283$) and large primes ($p \ge 283$).
\subsection{Small Primes Product ($P_{\mathrm{small}}$)}
For $p < 283$, $\omega_Q(p) = 0$. The partial product is:
\[
P_{\mathrm{small}} = \prod_{p < 283} \frac{p}{p-1}.
\]
Direct calculation over the 60 primes less than 283 yields:
\[
P_{\mathrm{small}} \approx 10.19.
\]
(Note: This value is consistent with Mertens' Theorem approximation $e^\gamma \log(283) \approx 10.05$).
\subsection{Large Primes Product ($P_{\mathrm{large}}$)}
For $p \ge 283$, non-trivial terms occur only when $p \equiv 1 \pmod{47}$.
\[
P_{\mathrm{large}} = \prod_{p \ge 283} \frac{1-\omega_Q(p)/p}{1-1/p}.
\]
This product converges conditionally. Numerical evaluation of the total product $C_Q$ up to large limits yields:
\begin{itemize}
    \item $X = 10^5$: $C_Q \approx 8.70$
    \item $X = 10^7$: $C_Q \approx 8.68$
\end{itemize}
We adopt the best estimate $C_Q \approx 8.68$.
\section{Asymptotic Prediction vs.\ Data}
The Bateman--Horn asymptotic prediction for the prime counting
function of $Q(n)$ is:
\[
\pi_Q(x) \;\sim\; \frac{C_Q}{46}\,\operatorname{Li}(x)
\;\approx\; \frac{8.68}{46}\,\operatorname{Li}(x)
\;\approx\; 0.1887\,\operatorname{Li}(x).
\]
Table~\ref{tab:comparison} compares the observed prime counts with
our prediction.
The observed values were computed using the Miller--Rabin primality
test (with deterministic bases for this range) and verified
independently.

\begin{table}[h]
\centering
\caption{Predicted vs.\ Observed Prime Counts for $Q(n)$.
Prediction uses $C_Q=8.68$.}
\label{tab:comparison}
\begin{tabular}{crrc}
\toprule
$x$ & Observed $\pi_Q(x)$ & Predicted $\frac{C_Q}{46}\operatorname{Li}(x)$ & Relative Error \\
\midrule
$10{,}000$ & 232 & 235 & $-1.3\%$ \\
$20{,}000$ & 429 & 432 & $-0.7\%$ \\
\bottomrule
\end{tabular}
\end{table}

\begin{remark}
The agreement to within $1{-}2\%$ at these small ranges is
consistent with expectations for the Bateman--Horn conjecture,
which is asymptotic.
The slight overestimate by the prediction is consistent with the
usual lower-order correction terms.
The verification range is limited to $x \le 20{,}000$ due to the
rapid growth of $Q(n)$: at $n=20{,}000$,
$Q(n) \approx n^{46} \approx 10^{197}$, requiring substantial
computational resources for primality testing.
Large-scale verification up to $n = 2\times 10^9$
($15.4$ million $Q$-primes) is reported in a companion paper.
\end{remark}
\section{Conclusion}
We have rigorously computed the Bateman--Horn constant for the Titan
polynomial $Q(n)=n^{47}-(n-1)^{47}$.
The value $C_Q \approx 8.68$ reflects the sequence's strong bias
towards primality due to the absence of small prime factors:
all~$60$ primes below~$283$ contribute a factor
$p/(p-1) > 1$ to the Euler product, while only the sparse set of
splitting primes $p\equiv 1\pmod{47}$ contributes factors
$(1-46/p)/(1-1/p) < 1$.
The resulting enhancement means that prime values of $Q(n)$ occur
approximately $8.7$ times more frequently than for a generic
degree-$46$ polynomial.
This quantitative analysis complements the algebraic and analytic
results established in companion papers, providing a complete
picture of the prime distribution of this cyclotomic norm form.

\medskip
\noindent\textbf{Data availability.}
The \LaTeX{} source, verification scripts, and supplementary data
are available at:
\begin{center}
\url{https://github.com/Ruqing1963/Q47-BatemanHorn-Constant}
\end{center}

\begin{thebibliography}{9}

\bibitem{ChenTitan}
R.~Chen,
\textit{Prime Values of a Cyclotomic Norm Polynomial and a
Conjectural Bounded Gap Phenomenon},
Preprint (2026),
\url{https://zenodo.org/records/18521551}.

\bibitem{BatemanHorn}
P.\,T.~Bateman and R.\,A.~Horn,
\textit{A heuristic asymptotic formula concerning the distribution
of prime numbers},
Math.\ Comp.\ \textbf{16} (1962), 363--367.

\end{thebibliography}
\end{document}
